% !TeX program = xelatex
%
%% retrotype-guide.tex
%% Copyright 2024 Vladimir Arabadzhi
%
% This work may be distributed and/or modified under the
% conditions of the LaTeX Project Public License, either version 1.3
% of this license or (at your option) any later version.
% The latest version of this license is in
%   https://www.latex-project.org/lppl.txt
% and version 1.3c or later is part of all distributions of LaTeX
% version 2008 or later.
%
% This work has the LPPL maintenance status `author-maintained'.
%
% The Author and Maintainer of this work is Vladimir Arabadzhi.
%
% This work consists of the following files:
%   retrotype.sty,
%   retrotype-guide.tex,
%   TT2020.fontspec.

\documentclass[english,12pt,openany,letterpaper]{book}
\usepackage{babel}
\usepackage{retrotype}
\usepackage{lipsum}
\usepackage{adjustbox}
\usepackage{layouts}

% HYPERSETUP

\usepackage{hyperref}
\usepackage{bookmark}

\def\title{Retrotype Guide}
\def\author{Vladar}
\def\license{LaTeX Project Public License 1.3c}
\def\licenselink{https://www.latex-project.org/lppl.txt}
\def\rtVer{0.2-devel}
\def\rtDate{2024-08-20}

\hypersetup{%
	pdfborderstyle={/S/U/W 1},% underlined hyperlinks
	bookmarksdepth=5,% paragraph level depth bookmarks
	pdftitle={\title},
	pdfauthor={\author},
	%pdfkeywords={\keywords},
}

\begin{document}

\addBg{\pageTiles{img/grain.jpg}}
\addBg{\pageFill[0.2]{cyan}}

\begin{titlepage}

\begin{verbatim}
________________________________________________________________________
             _____      _             _                                 
            |  __ \    | |           | |                                
            | |__) |___| |_ _ __ ___ | |_ _   _ _ __   ___              
            |  _  // _ \ __| '__/ _ \| __| | | | '_ \ / _ \             
            | | \ \  __/ |_| | | (_) | |_| |_| | |_) |  __/             
            |_|  \_\___|\__|_|  \___/ \__|\__, | .__/ \___|             
                                           __/ | |
__________________________________________|___/|_|______________________
\end{verbatim}

{\hfill v\rtVer~(\rtDate)}

\vfill

\begin{adjustbox}{center}
	\BVerbatimInput{img/dragon.txt}
\end{adjustbox}

\vfill
\vfill

{\hfill\license\hfill}

{\hfill\href{\licenselink}{(\licenselink)}\hfill}

{\hfill{}by \author{} (\href{https://vladar.bearblog.dev}{vladar.bearblog.dev})\hfill}

\end{titlepage}

%\smalltitles

\toc[3]

\chapter{Retrotype}

\clearBg
\clearFg

\rightbox[\baselineskip]{{\bfseries\itshape \rtDate}}

\section{The Objective}

Retrotype aims to replicate the look of old typewriter-made documents while keeping as many benefits of LaTeX as possible.


\section{The Implementation}

The most important part of achieving the objective is an appropriate font. \textbf{TT2020 font} (\href{https://copypaste.wtf/TT2020/}{copypaste.wtf/TT2020}) fits the goal perfectly and has 10 variants of each glyph to imitate the natural randomness of a real typewriter. The default font family is also set to monospace.

\tab To use this custom font, \textbf{XeTeX} (\href{https://xetex.sourceforge.net/}{xetex.sourceforge.net}) is required due to its \href{https://ctan.org/pkg/fontspec}{fontspec} capabilities. If you use \href{https://www.texstudio.org/}{TeXstudio}, you just need to have these lines at the start of your project:

\begin{tabbox}[8][8]
\begin{textbox}
~\% !TeX program = xelatex

~\bs documentclass[english,12pt,openany,letterpaper]\{book\}

~\bs usepackage\{babel\}

~\bs usepackage\{retrotype\}
\end{textbox}
\end{tabbox}

\section{Roadmap}

\begin{itemize}
	\item v0.1 \dash\ initial public testing release;
	\item v0.2 \dash\ \textit{work in progress};
	\item improving existing features and adding new ones if needed;
	\item v0.9 \dash\ all features complete, debugging and testing;
	\item v1.0 \dash\ release.
\end{itemize}


\section{Links}

\begin{itemize}
	\item \href{https://github.com/Vladar4/retrotype}{github.com/Vladar4/retrotype} \dash\ project's GitHub page;
	\item \href{https://vladar.bearblog.dev}{vladar.bearblog.dev} \dash\ author's blog.
\end{itemize}


\paragraph{Resources used:}
\begin{itemize}
	\item \href{https://copypaste.wtf/TT2020/}{copypaste.wtf/TT2020} \dash\ TT2020 typewriter font (\href{https://openfontlicense.org/}{OFL});
	\item \href{https://textkool.com/en/ascii-art-generator?font=Big&text=Retrotype}{textkool.com/en/ascii-art-generator?font=Big} \dash\ title text art;
	\item \href{https://www.asciiart.eu/mythology/dragons}{www.asciiart.eu/mythology/dragons} \dash\ dragon cover art;
	\item \href{https://www.publicdomainpictures.net/en/view-image.php?image=166303}{www.publicdomainpictures.net} \dash\ typewriter image;
	\item \href{https://lostandtaken.com/}{lostandtaken.com} \dash\ paper textures;
	\item \href{https://archive.org/details/Artyping}{archive.org/details/Artyping} \dash\  ``Artyping'' book by Julius Nelson.
\end{itemize}

\break


\section{Page Layouts}

While you can set up any custom layouts using the \bs newgeometry command (see the \href{https://www.ctan.org/pkg/geometry}{geometry} package), the following setups were fine-tuned:
\begin{itemize}
	\item Letter paper (8.5 x 11 in)
	\begin{itemize}
		\item 10pt font \dash\ 86 characters per line, 51 lines per page
		\item 11pt font \dash\ 80 characters per line, 45 lines per page
		\item 12pt font \dash\ 72 characters per line, 42 lines per page
	\end{itemize}
	\item Legal paper (8.5 x 14 in)
	\begin{itemize}
		\item 10pt font \dash\ 86 characters per line, 65 lines per page
		\item 80pt font \dash\ 80 characters per line, 58 lines per page
		\item 12pt font \dash\ 72 characters per line, 54 lines per page
	\end{itemize}
	\item Executive paper (7.25 x 10.5 in)
	\begin{itemize}
		\item 10pt font \dash\ 74 characters per line, 49 lines per page
		\item 11pt font \dash\ 68 characters per line, 43 lines per page
		\item 12pt font \dash\ 60 characters per line, 40 lines per page
	\end{itemize}
	\item A4 paper (210 x 297 mm)
	\begin{itemize}
		\item 10pt font \dash\ 84 characters per line, 55 lines per page
		\item 11pt font \dash\ 76 characters per line, 48 lines per page
		\item 12pt font \dash\ 70 characters per line, 45 lines per page
	\end{itemize}
	\item A5 paper (148 x 210 mm)
	\begin{itemize}
		\item 10pt font \dash\ 59 characters per line, 38 lines per page
		\item 11pt font \dash\ 54 characters per line, 34 lines per page
		\item 12pt font \dash\ 49 characters per line, 32 lines per page
	\end{itemize}
\end{itemize}


\subsection{Current Dimensions}

\newcommand{\curdim}[1]{\printinunitsof{pt}\prntlen{#1} & \printinunitsof{pc}\prntlen{#1} & \printinunitsof{mm}\prntlen{#1} & \printinunitsof{in}\prntlen{#1}}

\begin{retrotable}[lLLLL]
	Page width & \curdim{\pdfpagewidth} \\
	Page height & \curdim{\pdfpageheight} \\
	Text width & \curdim{\textwidth} \\
	Char width & \curdim{\charwidth} \\
	Baseline skip & \curdim{\baselineskip} \\
	Page layout & \pagelayout \\
	Font size & \docfontsizeInt \\
	Chars per line & \cpl \\
	Lines per page & \lpp \\
\end{retrotable}

~Page/Text width ratio:~ \pgfmathparse{\pdfpagewidth/\textwidth}\pgfmathresult

~Page/Text height ratio: \pgfmathparse{\pdfpageheight/\textheight}\pgfmathresult


\paragraph{Tabs}

tab~1~~~2~~~3~~~4~~~5~~~6~~~7~~~8~~~9~~~10~~11~~12~~13~~14~~15~~16~~17
v...v...v...v...v...v...v...v...v...v...v...v...v...v...v...v...v...v...
0...4...8...12..16..20..24..28..32..36..40..44..48..52..56..60..64..68..

\break

\section{Package Reference}

\subsection{New Lengths}

\subparagraph{\bs charwidth} \dash\ character width (currently \the\charwidth).

\subparagraph{\bs docfontsize} \dash\ document font size dimension (currently \the\docfontsize).

\subparagraph{\bs docfontsizeInt} \dash\ returns document font size (currently \docfontsizeInt).

\subparagraph{\bs tabwidth} \dash\ width of a tab (currently \the\tabwidth), see \textbf{Tabs} below. Changed automatically through the \bs setTabPositions call.


\subsection{New Symbols}

\subparagraph{\bs dash} \dash\ dash (\dash).

\subparagraph{\bs odash} \dash\ overline dash (\odash).

\subparagraph{\bs tld} \dash\ tilde (\tld).

\subparagraph{\bs bs} \dash\ backslash (\bs).

\subparagraph{\bs thelinechar} \dash\ the default line character (\bs dash by default).


\subsection{Pages}

\subparagraph{\bs pagelayout} \dash\ document page layout
\\\tab(Letter, Legal, Executive, A4, A5, undefined).
\subparagraph{\bs defaultGeometry} \dash\ reset to page geometry. Called automatically.
\subparagraph{\bs cpl} \dash\ characters per line (currently \cpl).

\subparagraph{\bs lpp} \dash\ lines per page (currently \lpp).

\subparagraph{\bs cleartoleftpage} \dash\ clear to left (even) page (new spread).

\subparagraph{\bs cleartorightpage} \dash\ clear to right (odd) page.


\subsection{Tabs}

\subparagraph{\bs tab} \dash\ go to the next tab in the current line.

\subparagraph{\bs tabn\{n\}} \dash\ skip to the \textbf{n}-th tab, counting from the current position.

\subparagraph{\bs tabspace} \dash\ inserts horizontal space equal to the current \bs tabwidth.

\subparagraph{\bs tabchars} \dash\ tab distance in characters (\tabchars). Changed automatically through the \bs setTabPositions call.

\subparagraph{\bs tabcharsDefault} \dash\ the default tab distance in characters (\tabcharsDefault),\\
used in \bs defaultTabPositions call.

\subparagraph{\bs defaultTabPositions} \dash\ reset \bs TabPositions to the default ones.

\subparagraph{\bs setTabPositions\{n\}} \dash\ changes distance between tab stops to \textbf{n}~characters wide. Also updates \bs tabchars and \bs tabwidth accordingly.

\subparagraph{\bs setCustomTabPositions\{list\}} \dash\ takes a \textbf{list} of comma-separated distances in \bs charwidth amounts to use as tab stops. Does \textbf{\ul{NOT}} update \bs tabchars and \bs tabwidth since the tabs can be distanced unequally.


\break


\subsection{Repeats and Lines}

\subparagraph{\bs rep\{chars\}\{n\}} \dash\ repeat the \textbf{chars} \textbf{n} times.

\subparagraph{\bs linechars[chars]\{n\}} \dash\ make a line by repeating the \textbf{chars} \textbf{n} times
\\\tab(\textbf{chars} = \bs thelinechar by default).

\subparagraph{\bs linecharw[chars]\{width\}} \dash\ repeat the \textbf{chars} for the given \textbf{width}
\\\tab(\textbf{chars} = \bs thelinechar by default).

\subparagraph{\bs linefill[chars]} \dash\ fill the whole line with \textbf{chars}
\\\tab(\textbf{chars} = \bs thelinechar by default).

\subparagraph{\bs linegoal} \dash\ the length of the remainder of the line
\\\tab(see the \href{https://www.ctan.org/pkg/linegoal}{linegoal} package).

\subparagraph{\bs strfit\{chars\}} \dash\ return int number of how many times the \textbf{chars} will fit into the remaining space on the line.

\break

\paragraph{Examples}

\bs rep\{Repeat \}\{10\}

\rep{Repeat }{10}

\skipline

\bs linechars[A]\{16\}

\linechars[A]{16}

\skipline

\bs linecharw[*]\{0.5\bs linewidth\}

\linecharw[*]{0.5\linewidth}

\skipline

\bs linefill

\linefill

\skipline

\bs linefill[-]

\linefill[-]

\skipline

\bs the\bs linegoal \dash\ the remaining space is \linegoal.

\skipline

\bs strfit\{abc\} \dash\ You can fit the ``abc'' text this much times: \strfit{abc}.


\break


\subsection{Stackengine}

Retrotype uses some capabilities of the \href{https://www.ctan.org/pkg/stackengine}{stackengine} package to simulate the overprinting of one character on top of another \dash\ a typewriting technique for creating new characters and graphic design elements.

\skipline

\subparagraph{\bs overprint\{a\}\{chars\}} \dash\ same as

\subparagraph{\tabspace\bs op\{a\}\{chars\}} \dash\ overprint \textbf{a} on top of all \textbf{chars}.

\subparagraph{\bs overprintw[width]\{a\}} \dash\ same as

\subparagraph{\tabspace\bs opw[width]\{a\}} \dash\ overprint \textbf{a} for the given \textbf{width} on the line, but keep printing from the starting point (\textbf{width} = \bs linegoal by default).

\skipline

\subparagraph{\bs underline\{chars\}} \dash\ same as

\subparagraph{\tabspace\bs ul\{chars\}} \dash\ underline given \textbf{chars}.

\subparagraph{\bs underlinefill} \dash\ same as

\subparagraph{\tabspace\bs ulf} \dash\ underline the rest of the line, but keep printing from the starting point.

\skipline

\subparagraph{\bs overline\{chars\}} \dash\ same as

\subparagraph{\tabspace\bs ol\{chars\}} \dash\ overline given \textbf{chars}.

\subparagraph{\bs overlinefill} \dash\ same as

\subparagraph{\tabspace\bs olf} \dash\ overline the rest of the line, but keep printing from the starting point.

\skipline

\subparagraph{\bs strikeout\{chars\}} \dash\ same as

\subparagraph{\tabspace\bs so\{chars\}} \dash\ strikeout given \textbf{chars}.

\subparagraph{\bs strikeoutfill} \dash\ same as

\subparagraph{\tabspace\bs sof} \dash\ strikeout the rest of the line, but keep printing from the starting point.

\break

\paragraph{Examples}

\vspace*{-2ex}

\begin{multicols}{3}
\begin{itemize}
	\item \textbf{\bs textbf\{\ldots\}}
	\item {\bfseries \{\bs bfseries \ldots\}}
	\item \strong{\bs strong\{\ldots\}}
	\item \textsl{\bs textsl\{\ldots\}}
	\item {\slshape \{\bs slshape \ldots\}}
	\item \textit{\bs textit\{\ldots\}}
	\item {\itshape \{\bs itshape \ldots\}}
	\item \emph{\bs emph\{\ldots\}}
	\item {\em \{\bs em \ldots\}}
\end{itemize}
\end{multicols}

\begin{itemize}
	\item \op{\textbar}{\bs op\{\bs textbar\}\{\ldots\}} \tab \bs op\{\bs textbar\}\{\ldots\} \tab \textit{(\bs overprint)}
	\item \opw{\textbar}\bs opw\{\bs textbar\}\ldots \tabn{7} \textit{(\bs overprintw)}
	\item \ul{\bs ul\{\ldots\}} \tab\bs ul\{\ldots\} \tab \textit{(\bs underline)}
	\item \ulf \bs ulf \ldots \tabn{3} \textit{(\bs underlinefill)}
	\item \ol{\bs ol\{\ldots\}} \tab\bs ol\{\ldots\} \tab \textit{(\bs overline)}
	\item \olf \bs olf \ldots \tabn{3} \textit{(\bs overlinefill)}
	\item \so{\bs so\{\ldots\}} \tab\bs so\{\ldots\} \tab \textit{(\bs strikeout)}
	\item \sof \bs sof \ldots \tabn{3} \textit{(\bs strikeoutfill)}
\end{itemize}

\skipline

\bs linefill[\bs op\{<\}\{>\}]

\linefill[\op{<}{>}]

\skipline

\bs linefill[\bs ol\{\bs ul\{\tld\}\}/]

\linefill[\ol{\ul{~}}/]

\skipline

\bs linefill[\bs ol\{\bs ul\{\bs op\{/\}\{\bs bs\}\}\}\}]

\linefill[{\ol{\ul{\op{/}{\bs}}}}]

\skipline

\bs linefill[\bs op\{\bs dash\}\{\bs op\{(\}\{)\}\}]

\linefill[\op{\dash}{\op{(}{)}}]

\skipline

\bs linefill[\bs ol\{\bs bs\bs\ /\}]\\
\bs vspace*\{-1ex\}\\
\bs linefill[\bs so\{\bs\ \}\bs ol\{\bs ul\{|\}\}\bs so\{\bs\ \}]\\
\bs vspace*\{-1ex\}\\
\bs linefill[\bs ul\{/\bs \ \bs bs\}]\\

\linefill[\ol{\bs\ /}]
\vspace*{-1ex}
\linefill[\so{\ }\ol{\ul{|}}\so{\ }]
\vspace*{-1ex}
\linefill[\ul{/\ \bs}]

\vfill
\break


\subsection{Formatting}

\subparagraph{\bs skipline[height]} \dash\ insert a vertical space (\bs baselineskip by default).

\skipline

\subparagraph{\bs mirrorh\{text\}} \dash\ mirror \textbf{text} horizontally.

\subparagraph{\bs mirrorv\{text\}} \dash\ mirror \textbf{text} vertically.

\subparagraph{\bs mirrorboth\{text\}} \dash\ mirror \textbf{text} horizontally and vertically.

\skipline

\subparagraph{\bs rightbox[height]\{text\}} \dash\ a right-aligned line of  \textbf{text} raised by \bs baselineskip + height. The default height is \bs parskip to put the text on a \bs paragraph\{\} line above.

\skipline

\subparagraph{\bs tabpar[l][r]\{text\}} \dash\ a single paragraph indented from the \textbf{l}eft and \textbf{r}ight by \textbf{l} and \textbf{r} characters (\textbf{l} = \bs tabchars (\tabchars), \textbf{r} = 0 by default).

\subparagraph{\{tabbox\}[l][r]} environment \dash\ \textbf{text} indented from the \textbf{l}eft and \textbf{r}ight by \textbf{l} and \textbf{r} characters respectively (\textbf{l} = \bs tabchars (\tabchars) and \textbf{r} = 0 by default). \textbf{\{tabbox\}*[l][r]} version switches the values between odd and even pages.

\skipline

\subparagraph{\bs header[chars][format]\{text\}} \dash\ \textbf{text} with \textbf{chars} filling the line
\\(\textbf{chars} = \bs\_, \textbf{format} = \bs bfseries by default). Not a sectioning element.

\subparagraph{\bs uheader[chars][format]\{text\}} \dash\ same as \textbf{\bs header}, with \textbf{text} underlined.

\skipline

\subparagraph{\{headerbox\}[format]\{header\}[top][bottom]} environment \dash\ text box with a~\textbf{header}, surrounded by lines (\textbf{top} = \bs\_, \textbf{bottom} = \bs odash by default).

\subparagraph{\{uheaderbox\}[format]\{header\}[top][bottom]} environment \dash\ same as \textbf{\{headerbox\}}, with \textbf{header} underlined.

\skipline

\subparagraph{\{textbox\}[top][bottom]} environment \dash\ text box surrounded by top and bottom lines (\textbf{top} = \bs\_ and \textbf{bottom} = \bs odash by default).

\break

\paragraph{Examples}

\rightbox{\bs rightbox\{Right-aligned text.\}}

\mirrorh{\bs mirrorh} \tab \bs mirrorh\{\bs bs mirrorh\}

\mirrorv{\bs mirrorv} \tab \bs mirrorv\{\bs bs mirrorv\}

\mirrorboth{\bs mirrorboth} \bs mirrorboth\{\bs bs mirrorboth\}

\skipline

\bs tabpar[6][6]\{\bs lipsum[66]\}
\\\textbf{or} \bs begin\{tabbox\}[6][6] \bs lipsum[66] \bs end\{tabbox\}

\tabpar[6][6]{\lipsum[66]}

\skipline

\header{header title}

\bs header\{header title\}

\skipline

\uheader{uheader title}

\bs uheader\{header title\}

\skipline

%\begin{headerbox}[\normalfont\ul]{{\textbf{header box title}}}
\begin{uheaderbox}{\textbf{box title}}
	\bs begin\{headerbox\}[\bs normalfont\bs ul]\{\{\bs textbf\{box title\}\}\} ... \bs end\{headerbox\}
	\\\textbf{or}
	\bs begin\{uheaderbox\}\{box title\} ... \bs end\{uheaderbox\}
\end{uheaderbox}

\begin{textbox}
	\bs begin\{textbox\} ... \bs end\{textbox\}
\end{textbox}

\break


\subsection{Retrotable}

A modified \href{https://www.ctan.org/pkg/tabularx}{tabularx} environment.

\begin{tabbox}[4][4]
	\begin{headerbox}{NOTE:}
		Work in progress.
		
		There should be an easier way to implement the horizontal lines than using the \bs multicolumn hack (see the \textbf{Examples}).
	\end{headerbox}
\end{tabbox}

\bs begin\{retrotable\}[columns][width]

\tab ...

\bs end\{retrotable\}

(width is \bs linewidth by default)

\skipline

\paragraph{New column types}

\begin{itemize}
	\item L left-aligned expandable column (same as X from tabularx)
	\item C centered expandable column
	\item R right-aligned expandable column
	\item | vertical separator
	\item \_ top line (underline: \_ )
	\item - middle line (dash: \dash{} )
	\item \^{} bottom line (overline: \ol{~} )
	\item * asterisk
	\item \~{} empty space
\end{itemize}

\break

\paragraph{Examples}

\begin{verbatim}
\begin{retrotable}[|C|L|R|][0.5\linewidth]
    \multicolumn{3}{_l_}{\linecharw[\_]{0.5\linewidth}}\\
    Center & Left & Right \\
    1 & 2 & 2 \\
    4 & 5 & 6 \\
    \multicolumn{3}{^l^}{\linecharw[\odash]{0.5\linewidth}}\\
\end{retrotable}
\end{verbatim}

\begin{retrotable}[|C|L|R|][0.5\linewidth]
	\multicolumn{3}{_l_}{\linecharw[\_]{0.5\linewidth}}\\
	Center & Left & Right \\
	1 & 2 & 2 \\
	4 & 5 & 6 \\
	\multicolumn{3}{^l^}{\linecharw[\odash]{0.5\linewidth}}\\
\end{retrotable}


\vfill
\break


\subsection{Table of Contents}

\begin{tabbox}[4][4]
	\begin{headerbox}{multitoc}
		To make your table of contents two-column, use this package:
		
		\bs usepackage[toc]\{multitoc\}
	\end{headerbox}
\end{tabbox}

\subparagraph{\bs toc[n]} \dash\ print table of contents to the \textbf{n}-th depth.


\subsection{Titles}

\subparagraph{\bs chapterx} \dash\ unnumbered chapter with a TOC entry.

\subparagraph{\bs largetitles} \dash\ set varying font sizes for the titles (the default).

\subparagraph{\bs smalltitles} \dash\ set all titles' font size to \bs normalsize.

\linefill
\vspace*{-\baselineskip}

\begin{multicols}{2}
	\titleclass{\chapter}{straight}
	\largetitles{}
	\chapter*{Large Titles}
	\section*{Section}
	\subsection*{Subsection}
	\subsubsection*{Subsubsection}
	\paragraph*{Paragraph}
	\paragraph*{Subparagraph}
	
	\columnbreak
	
	\smalltitles{}
	\chapter*{Small Titles}
	\section*{Section}
	\subsection*{Subsection}
	\subsubsection*{Subsubsection}
	\paragraph*{Paragraph}
	\paragraph*{Subparagraph}
	\titleclass{\chapter}{top}
\end{multicols}

\vspace*{-4ex}
\linefill

\subsubsection{Title Spacing}

Current skip lengths before and after each sectioning level are available through \bs b<title>skip and \bs a<title>skip macros:

\vspace*{-2ex}
\begin{multicols}{2}
	\paragraph{Before}
	\begin{itemize}
		\item \bs bchapskip
		\item \bs bsecskip
		\item \bs bsubsecskip
		\item \bs bsubsubsecskip
		\item \bs bparskip
		\item \bs bsubparskip
	\end{itemize}
	\columnbreak
	\paragraph{After}
	\begin{itemize}
		\item \bs achapskip
		\item \bs asecskip
		\item \bs asubsecskip
		\item \bs asubsubsecskip
		\item \bs aparskip
		\item \bs asubparskip
	\end{itemize}
\end{multicols}


\subsection{Verbatim}

\subparagraph{\bs VerbatimInput\{file\}, \bs BVerbatimInput\{file\}, \bs LVerbatimInput\{file\}} \dash\ include text \textbf{file} verbatim (normal, TeX box, LR mode).

\subparagraph{\{Verbatim\}, \{BVerbatim\}, \{LVerbatim\}} \dash\ verbatim text environments.

\subparagraph{\bs renewcommand\{\bs FancyVerbFormatLine\}[1]\{\#1\}} \dash\ execute arbitrary code for each line of verbatim text.

\subparagraph{\bs fvset\{options\}} \dash\ set comma-separated verbatim text \textbf{options}.

All commands of the \href{https://www.ctan.org/pkg/fancyvrb}{fancyvrb} package are also available by default.

\vfill
\cleartorightpage

\subsection{Background and Foreground}

\subparagraph{\bs addBg\{code\}} \dash\ add \textbf{code} to the background picture.

\subparagraph{\bs addFg\{code\}} \dash\ add \textbf{code} to the foreground picture.

\subparagraph{\bs clearBg} \dash\ clear the background picture.

\subparagraph{\bs clearFg} \dash\ clear the foreground picture.

\skipline

\paragraph{Commands}

\subparagraph{\bs pageFill[opacity][blend]\{color\}} \dash\ fill the page with the given color:
\begin{itemize}
	\item \textbf{opacity} \dash\ color opacity (1.0 by default);
	\item \textbf{blend} \dash\ color blend mode (normal by default);
	\item \textbf{color} \dash\ color to fill the page with.
\end{itemize}

\subparagraph{\bs pageCenter[opacity][blend]\{file\}[scale]} \dash\ image centered on the page:
\begin{itemize}
	\item \textbf{opacity} \dash\ image opacity (1.0 by default);
	\item \textbf{blend} \dash\ image blend mode (normal by default);
	\item \textbf{file} \dash\ image file name;
	\item \textbf{scale} \dash\ image scale (1.0 by default).
\end{itemize}

\subparagraph{\bs pageStretch[opacity][blend]\{file\}[ratio]} \dash\ image stretched over the page:
\begin{itemize}
	\item \textbf{opacity} \dash\ image opacity (1.0 by default);
	\item \textbf{blend} \dash\ image blend mode (normal by default);
	\item \textbf{file} \dash\ image file name;
	\item \textbf{ratio} \dash\ keep aspect ratio (default = 1, make 0 to ignore).
\end{itemize}

\subparagraph{\bs pageTiles[opacity][blend]\{file\}[scale]} \dash\ tile the image over the page:
\begin{itemize}
	\item \textbf{opacity} \dash\ image opacity (1.0 by default);
	\item \textbf{blend} \dash\ image blend mode (normal by default);
	\item \textbf{file} \dash\ image file name;
	\item \textbf{scale} \dash\ image scale (1.0 by default).
\end{itemize}

\skipline

\begin{headerbox}{Blend modes}
\vspace*{-\baselineskip}\noindent\begin{multicols}{4}
	\begin{enumerate}
		\item normal
		\item multiply
		\item screen
		\item overlay
		\item darken
		\item lighten
		\item color dodge
		\item color burn
		\item hard light
		\item soft light
		\item difference
		\item exclusion
		\item hue
		\item saturation
		\item color
		\item luminosity
	\end{enumerate}
\end{multicols}
\vspace*{0.5ex-2\baselineskip}~
\end{headerbox}

\skipline

\paragraph{Examples}

\bs addBg\{\bs pageTiles\{img/grain.jpg\}\}

\bs addBg\{\bs pageFill[0.2]\{cyan\}\}

\bs addFg\{\bs pageCenter[0.15][multiply]\{img/typewriter.png\}[0.5]\}

\addBg{\pageTiles{img/grain.jpg}}
\addBg{\pageFill[0.2]{cyan}}
\addFg{\pageCenter[0.15][multiply]{img/typewriter.png}[0.5]}


\twocolumn

\clearFg

\chapter{Demonstration}

\subsection{1. Bridge}

An old stone bridge over the moat. Moss and lichen growing between crumbling stonework. A large stone gatehouse stands at the opposing side of the bridge.

\subsection{2. Gatehouse}

The main entrance to the castle \dash\ two square towers with a gate and portcullis between them. Some old crates and barrels are piled near the entrance. On either side of the gate, there are small arrow slits in the tower walls.

\tab A group of goblins is hiding behind crates and barrels near the gate. The goblins are armed with short swords and small shields and are on the lookout for intruders.


\header{Goblins (2d4)}
AC~5; HD~1-1~(S); hp~4~each; MV~90' (30'); \#AT~1 short sword; D~1d6; Save~NM; ML~7; Int~9; AL~C; XP~5; THAC0~19. Treasure: 2d6 ep each.

\subsection{3. Gatehouse Towers}

The gatehouse towers' interior is dimly lit, with narrow stone staircases winding up the square towers to the top.

\subsection{4. Castle Wall}

A sturdy stone wall surrounds the entire castle. Four square towers stand at the corners. The moat encircles the wall, making its effective height even taller.

\break

\subsection{5. Castle Towers}

Square towers at the corners of the wall provide a clear view of the surrounding area. Heavy wooden doors open to the top of the castle wall. A wooden ladder leads to the top of each tower.

\skipline
\subsubsection{5a) North-West Tower}

A group of goblins is lounging inside the tower. They can easily be taken by surprise.

\header{Goblins (2d4)}
AC~6; HD 1-1~(S); hp~4~each; MV~90' (30'); \#AT~1 short sword; D~1d6; Save~NM; ML~7; Int~9; AL~C; XP~5; THAC0~19. Treasure: 2d6 ep each.

\skipline
\subsubsection{5b) North-East Tower}

Nothing of note inside.

\skipline
\subsubsection{5c) South-West Tower}

A group of orcs will attack the party on sight.

\header{Orcs (2d4)}
AC~6; HD~1~(M); hp~5~each; MV~120' (40'); \#AT~1 scimitar; D~1d8; Save~F1; ML~8; Int~7; AL C; XP~10; THAC0~19. Treasure: 3d8 cp each.

\skipline
\subsubsection{5d) South-East Tower}

The floor at the top of the tower is weak and has a 1-in-6 chance of collapsing if anyone steps on it. The risk of collapse increases by 1 per each additional human-sized creature standing there. The fall deals 1d6 damage.

\clearpage


\end{document}
